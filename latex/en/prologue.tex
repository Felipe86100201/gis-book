\chapter*{Pr�logo}

Hace ahora m�s de cinco a�os que se public� la primera versi�n de \emph{Sistemas de Informaci�n Geogr�fica}, un libro libre sobre fundamentos de SIG en espa�ol, y apenas unos meses desde que apareci� la segunda. El libro ha tenido una acogida excelente, y mi intenci�n es seguir manteni�ndolo actualizado en la medida que sea posible, reflejando los avances que, a buen seguro, van a producirse en el campo de los SIG.

Existe, no obstante, un obst�culo importante para que el libro alcance a todos los p�blicos: su tama�o. Por su completitud, y por la complejidad propia de la disciplina, el libro es un volumen de m�s de 800 p�ginas cargadas de detalle. La segunda versi�n se presenta en un �nico tomo, frente a los dos en que consist�a la primera, pero a�n as� sigue quedando como una obra de consulta demasiado extensa para leerse de principio a fin. Para el lector que comienza a introducirse en el ambito de los SIG y no busca especializarse, resulta un volumen intimidante y es, no hay duda, dif�cil de abordar.

Este libro intenta ser una alternativa a la obra completa, de tal forma que resulte m�s accesible para quienes desean tener una perspectiva global de la disciplina de los SIG, sin entrar en detalles demasiado espec�ficos. Es, basicamente, una versi�n resumida de aquel, pensada con la idea de usarse no como libro de consulta, sino como libro de lectura. Adem�s de ser m�s breve, se presenta en un formato m�s adecuado para esta clase de prop�sito, con algunas modificaciones en su enfoque y con menos contenido gr�fico.

He respetado en l�neas generales la estructura de los cap�tulos, de modo que es f�cil para el lector que desee profundizar en uno de ellos encontrar este en el libro completo. Desde ese punto de vista, puede entenderse este libro como una especie de <<�ndice>> de su hermano mayor, un �ndice, no obstante, prolijo y con suficiente informaci�n como ofrecer al lector una visi�n detallada del mundo de los SIG.

Este es tambi�n, por supuesto, un libro libre, que espero que progrese de una forma din�mica gracias a la contribuci�n de sus lectores. Si encuentras cualquier error o quieres colaborar en mejorar estas p�ginas, no dudes en escribirme a \texttt{volayaf@gmail.com}.