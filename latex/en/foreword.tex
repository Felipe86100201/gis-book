\chapter*{Foreword}

I first met Victor when we were working at the same company. During that time I learned a few interesting things about him. Like how he works at such a rapid pace that if you blink you might find that he's written a new plugin for QGIS or even that he's written a book like this one. Aside from these great qualities, the thing I most remember about him is that he helped direct me to the last packet of hot chocolate in the office kitchen, after a day full of meetings when I needed it the most. It's helpful things like that which make a difference to people. And in this book you will find so many helpful things, akin to that hot chocolate but for Geographic Information Systems (GIS), organized in a thoughtful manner which will help you get through that sometimes-long GIS slog.


This book is an excellent reference text regarding the history and basics of GIS. It includes clear examples of concepts illustrating choices the geospatial professional must make in design and layout and how those choices affect a map product. The reader can literally see how decisions about line, color, shape, and other qualities will render a map that is the most useful and the most aesthetic. It also includes important information about the various ways in which GIS data is obtained, how it is stored, and a great overview of GIS software.

The book begins with the history of GIS and proceeds into sections that discuss and define such topics as spatial analysis, data visualization, web mapping and data sources, among many others. I envision the book being used as a teaching tool, both in a formal setting and for self-learners. Additionally, for more experienced geospatial professionals, this book can be used in the initial ideation phase of creating a map, reminding us of the elements we need to consider and prioritize to meet the objectives for a particular map or analysis. It is really a digital pocket guide to GIS.

Victor is generously making his book available to all, free, for users. Knowing the hours of work that go into any book, I appreciate his attitude of community and contribution to the field of GIS. Learning and continually revisiting the fundamentals is paramount for success in our field. So pour yourself a good cup of hot chocolate and get started.  



\begin{flushright}
\textsc{Gretchen Peterson}\\
Co-author of QGIS Map Design
\end{flushright}