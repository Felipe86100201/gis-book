
\chapter{�What is a GIS?}

\pagestyle{fancy}

Most of the information that we use nowadays is georeferenced. That is, it is information to which a geographical position can be assigned, and it's thus information that has some ancillary information related to its location.


A \textbf{Geographical Information System} (GIS) is a tool to work with feoreferenced information. In particular a GIS is a system that allows the following operations:

\begin{itemize}
	\item \textbf{Reading, editing storing}, and, generally speaking, \textbf{managing} spatial data.
	\item \textbf{Analyzing} those data. That includes from simple queries to complex models, which can be performed using whether the \textbf{spatial component} of the data (the location of each value or element), the \textbf{thematic component} (the value or element itself), or both.
	\item Generating \textbf{documents} such as maps, reports, plots, etc.
\end{itemize}


A GIS is a step further from traditional maps. A map represents a rendering of a set of spatial data, and, while this rendering has a great importance within a GIS, is it but one of its many components. A GIS includes not only data and its rendering, but also all the operations that can be perform on it, which are part as well of the system.

A GIS is a flexible and versatile tool, and most disciplines today use GIS in one way or another. One of the main reasons for this is the integrative nature of GIS. The following ones are some of the main contexts in which GIS plays this integrative role.



\begin{itemize}
\item \textbf{GIS as a tool to integrate information}. A common link between most disciplines is that they study something which can be located. This allows to combine them and get results from a joint analysis. In this context, GIS provides the framework to which that information from different disciplines can be added and in which we can work with it. 

\item \textbf{GIS as a tool to integrate technologies}. A large part of the technologies that have appeared in the last years (and most likely of those that will appear in the near future) is based on using spatial information, and are connected to some extent to a GIS to extend their capabilities and their reach. Due to its central position in this group of technologies, GIS plays a central role in linking them and allowing them to establish a fluent comunication, center around its own functionalities.


\item \textbf{GIS as a tool to integrate technologies}. GIS functionalities cover a broad range of users, most of which would not have such a well-defined framework if it was not for GIS itself. As a consequence of that, there is  better coordination between them.

\item \textbf{GIS as a tool to integrate theoretical areas}}. We can understand GIS as the sum of two disciplines: geography and computer science. However, a more detailed analysis reveals that a GIS incorporates elements from many different scientific fields, such as those related to technology and data management (computer science, database design, digital image analysis), those that study the Earth from a physical point of view (geology, oceanography, ecology) or from a social and human one (antropology, geography, sociology), those that study human behaviour and understanding (psicology) or those that have themselves traditionally integrated knowledge from different fields, such as the already mentioned geography.

The term \textbf{geomatics}, derived from \emph{geography} e \emph{informatics}, is frequently used to refer to the array of scientific areas related to GIS.
\end{itemize}

With this, we have that a GIS is a system that integrates technology, informatics, people and geographical information, and whose main purpose is to capture, analize, store edit and visualize georeferenced data.


From a different point of view, a GIS can be considered as composed of five main elements:


\begin{itemize}
 \item \textbf{Data.} Data is needed for the rest of the components to make sense and be able to serve a given purpose. Geographical information, the core of GIS, lives in the data, and a detailed knowledge of the data that we use, its quality, its origin, its characteristics, and how to manage and store it is paramount to correctly understand GIS itself.

\item \textbf{Analysis.} Analysis is one of the main strengths of GIS, and one of the reasons why the first GIS were developed. Most GIS include analysis capabilities. They include formualtions that were already used with traditional cartography, others that existed but were not feasible to use without computers, and new approaches that were developed specifically after GIS appeared.

\item \textbf{Visualization}. All types of information can be represented graphically, which makes it easier to interpret it. In the particular case case of geographic information, visualizing it is not only a different way of working with that information, but indeed the main one, since it's the one we are more used to.

While maps are graphical entities, in a GIS we work with raw alphanumeric data. In order to have the same capabilities of a printed map, a GIS must be able to create visual representations from that data, including map-like ones.

The same cartographic principles that apply when designing a printed map are also valid when rendering geographic data within a GIS, and GIS users must be familiar with them. 

\item \textbf{Technology.} That includes both the GIS software and the hardware that runs it. Additional elements that are common when working with GIS data, such as peripherics user for data entering or for creating printed cartography, are also included in here.

\item \textbf{Organization.} This includes the elements that ensure a proper coordination between people, data and technology. As GIS gets more complex, managing the relations between its elements becomes more important.
\end{itemize}

In the following chapters, we will describe these elements in detail.




\pagestyle{empty}