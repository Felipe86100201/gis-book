
\chapter{History of GIS}


\pagestyle{fancy}

GIS has experienced a huge development since its early days. With the popularization of GIS technologies, and thanks to the help of all other disciplines that use GIS and rely on it, the field of GIS has been redefined and expanded, especially in the last years.

We can locate the origins of GIS in the sixties, when the first GIS applications appeared. The two main reasons for this were the \textbf{increasing need of geographical information} and the \textbf{appearance of the first computers}.

The theoretical foundation of GIS was laid a few years before, with the development of new approaches in the field of cartography, such as \textbf{quantitative cartography}, which seemed to predict the future needs that the use of computers and geographical data would bring.

The first relevant experience that combined computers and geography can be found in 1959, when Waldo Tobler defined the principles of a system called MIMO (map in--map out), with the purpose of applying computers to the field of cartography. He defined the basic ideas for creating, encoding, analyzing, and rendering geographical data within a computer system.

The first GIS was the CGIS (Canadian Geographical Information System). It was developed in Canada in the early sixties by Roger Tomlinson, who is popularly known as the ``father of GIS''.

In the mid-sixties, two applications, SYMAP and GRID, laid out the theoretical foundation for the analysis of \textbf{raster} and \textbf{vector} data, the two main approaches for encoding and storing geographical information (we will explain them in detail in the upcoming chapters). The main ideas for performing analysis in raster GIS were defined by Dana Tomlin with his \textbf{map algebra}.

During the sixties, the field of GIS starts developing itself from those seminal works. GIS is not anymore an experimental tool, and it starts to become and important part of the cartographic world.

From this moment, GIS evolves through several different periods, moving very fast thanks to the influence of many external factors. This evolution affects the discipline of GIS itself, the technology it involves, the data, and also the theories and techniques it is built on.

\section{The evolution of GIS as a discipline}

At first, GIS was just a combination of ideas from quantitative cartography, and the computer systems that existed at that time. It was basically the work of cartographers and geographers who tried to adapt their knowledge and their needs to a technology that looked promising. Since then, a large number of other disciplines have contributed to the field of GIS and their contributions are as important, or in some cases even more so, than those of cartography and geography.

More or less at the same time, society was becoming more concerned about the environment and the effect of human actions on it. This influenced GIS which was becoming a fundamental tool for all tasks related to environmental management (land-use planning, environmental monitoring, etc.), and boosted its development.

At the beginning of the seventies, once it was clear that GIS had a great future ahead, the field of GIS started to shape its identity and to become a solid discipline. The first conferences and symposiums about GIS took place and GIS was already included in University \emph{curricula}. Specialized journals and forums appeared in the eighties and helped spread GIS to a wider audience, The industry of GIS consolidated itself in the seventies. \textbf{ESRI} (Environmental Systems Research Institute), pioneer and current leader of the GIS market, was founded is 1969, and its products have played a key role in the popularization of GIS. The first open-source GIS, \textbf{GRASS} (Geographic Resources Analysis Support System), appeared in 1985.

The beginning of the 21st century marks a turning point in the history of GIS, as it reaches non-professional audiences. Cartography services such as \textbf{Google Maps} allow users with little or no technical GIS knowledge to interact with a GIS application and use it. \textbf{GPS navigators}, which include both analysis and rendering capabilities that come from GIS, are another good example of this.

\section{The evolution of technology}

The evolution of computers has affected GIS. Three are the main areas that have had a major influence in shaping GIS as we know it now.

\begin{itemize}
 \item \textbf{Graphical outputs}. The capabilities of computers to generate graphical outputs have greatly improved since their beginnings, and they are still evolving. GIS has followed this evolution closely, both for screen rendering and for the case of printed outputs.
\item \textbf{Data access and storage.} The size of GIS datasets has increased enormously, and using these large datasets would not be possible without the corresponding improvements in both data storage and data access.
\item \textbf{Data input}. In the early days of GIS, data were manually digitized. Nowadays, creating data that can be used in a GIS is a completely different process, and it uses specific hardware such as high-resolution scanners, or specific software such as the one used for automatic digitalization of pattern recognition based on images, all of which generate ready-to-use data.
\end{itemize}

Along with this, software has changed following the evolution of computers themselves, from mainframes to personal computers, and more recently, to other platforms such as tablets or mobile phones.

By the end of the eighties, cartography can be efficiently produced in personal computers, with a comparatively low cost, without the need of expensive and dedicated large mainframes.

Nowadays, the combination of positioning systems such as GPS with mobile platforms is playing and important role in the development of GIS, in areas such as data collection.

The Internet also changed GIS, much like it changed every other field, whether scientific or not. In 1993, \emph{Xerox PARC}, the first \textbf{map server} to distribute cartography over the Internet, was created. The first digital on-line atlas, the Canadian National Atlas, has been available since 1994. More recently, the ideas of the Web 2.0 are adapted to the field of GIS and contribute to the development of what is now known as \textbf{Web Mapping}.

\section{The evolution of data}

The first geographical datasets used in GIS contained just \textbf{scanned maps} and \textbf{digitized features} obtained from them. Since then, new data sources have been constantly appearing, with formats that are better adapted to GIS, and with GIS itself adapting to them as well. As a consequence of that, the amount, precision, and quality of data that is now available to be used in a GIS has dramatically increased.

The launching of the first \textbf{earth observation satellites} represents a key advance. The techniques that were already in use for aerial photography, developed mostly during the First World War (although the discipline goes back to the second half of the 19th century, when photos were taken from hot air balloons), are applied on a global scale when the first satellites are created. SPOT Image, the first commercial company to distribute satellite images that cover the entire globe, was created in 1982.

Positioning technologies are another important data source for GIS. In 1981, the GPS system became completely operative, and in 2001, its accuracy for civil use was increased.

As it happened with GIS software, digital geographical data becomes more popular and receives more attention. In 1976, the United States Geological Service (USGS) publishes its first \textbf{Digital Elevation Models} (DEM), in response to the high relevance that this type of data now had in the context of geographical analysis. In 2000, elevation data from the \emph{Shuttle Radar Topographic Mission}(SRTM) is released to the public, covering 80\% of the Earth's surface with a resolution of one arc second (about 30 meters).

The development of techniques such as \textbf{LiDAR}, which can be used to get elevation data with much more detail, opens a large array new possibilities for areas such as terrain analysis.

The evolution of data is not just technical, but also \textbf{social and organizational}. As the amount of data increases, it becomes clear that new strategies must be developed for managing those data. So-called \textbf{Spatial Data Infrastructures} are developed as a result of this. The most relevant of them is the United States National Spatial Data Infrastructure (NSDI), created in 1994. In Europe, the INSPIRE directive serves a similar purpose.

Many of these activities and developments follow the specifications set up by the \textbf{Open GIS Consortium} (OGC), and international consortium founded in 1994, which works to \textbf{homogenize and standardize} the use and distribution of geographical data.

\section{The evolution of theories and techniques}

Once the first GIS was implemented and could respond to the data management and analysis needs for which they were created, new techniques and approaches began to be developed. 

Spatial analysis is a comparatively recent field. In 1854, \textbf{John Snow} performed what is usually considered one of the first examples of analytical cartography, when he used a map to determine the source of a cholera outbreak in London.

In his book \emph{Design with Nature} (1969), Ian McHarg defined the basic ideas about  \textbf{map overlays}, which, as we will later see, are fundamental for the analysis and visualization of geographical data \textbf{layers} within a GIS.

Terrain analysis is another field that has experienced a huge qualitative change thanks to GIS. Traditional terrain analysis, mostly based on geology and geomorphological analysis, developed into a quantitative science focused on the morphometric analysis of relief.

Along with the analytical component, cartography also evolved in the context of GIS. In 1819, Pierre Charles Dupin created the fist \textbf{choropleth map}. With the arrival of GIS, this type of map will become very popular.

The advances in Computer-Assisted Design (CAD) applications and in-screen rendering techniques helped in defining a new discipline: computational geometry. GIS vector analysis is based on it.

\pagestyle{empty}
